%\input{|"curl -L 'https://virginia.box.com/shared/static/9120uqo655t0kwesti7t3exa7zsgbwdq.tex'"}
\include{article_header}
\title{Absence Makes the Ideal Points Sharper: Full-data IRT Models for Legislatures}
\usepackage{amsmath,amsthm, amssymb, latexsym}
\linespread{1.5}
\begin{document}
	
	\maketitle
	
	\begin{abstract}
		I put forward a Bayesian IRT model that can handle legislators absences as a separate category of data for determining legislator ideal points. The estimation uses the concept of a hurdle model to deflate the probabilities of legislator’s votes by the probability of absences. The model produces a single set of ideal points, but utilizes different parameters for bill absence points. Compared to existing approaches, this model tends to produce more moderate estimates of US Congresspeople’s ideal points because it can model roll-call votes where there are very few opposing votes. For parliamentary data, the model provides much more precise estimates, especially in legislatures with very high rates of absence. Additionally, the model can incorporate absentions as a middle category between yes and no votes for legislatures with high rates of abstentions.
	\end{abstract}
	
	Ideal point modeling of legislatures, and increasingly diverse kinds of social actors, has become an increasingly important part of empirical work in political science. However, most models of ideal points are based on binary outcomes that reflect yes and no votes (or positions) while all other actions are recorded as missing data. The argument for doing so is that the yes and no votes contain most of the information about legislator ideal points, and that incorporating other categories such as abstentions or absences would greatly complicate the estimation without adding much benefit. In relation to the first point, I present a model that uses Bayesian Markov Chain Monte Carlo (MCMC) estimation to handle the contingency inherent in the data. In relation to the second, I show in this paper that absences and abstentions do contain important information about legislator ideal points, a contention that has become more prominent in recent research.
	
	However, I depart from the discussion of ideal point models in this recent literature by treating absences and abstentions as additional data rather than as missing yes/no votes. This distinction, while somewhat technical, has important ramifications for modeling choices. If absences and abstentions represent missing yes/no votes, then the solution is to impute how a legislator would have voted if they had not either been absent or voted to abstain. Accomplishing this result is no easy feat because multiple imputation methods require that at some level the unobserved outcomes be random, or ignorable, relative to the observed outcomes (i.e., yes/no votes). Given that legislator behavior is usually strategic and not random, I argue that the requirements of multiple imputation are unlikely to be met without relying on additional parametric assumptions. 
	
	In this situation, the difficulty of multiple imputation is not a difficulty because the observed outcomes--absence and abstention--offer potential insight into ideal points. To this end, I have decided to incorporate absences and abstentions by adding them as outcomes within a standard item-response theory (IRT) ideal point model. While the ensuing model is more complicated than a traditional ideal point model, the model estimates a single set of ideal points, which means that the ideal points will be more precisely estimated than with a model that uses only yes or no votes. By treating absences and absentions as additional observations of legislator behavior, rather than as a statistical problem to be overcome, ideal point models can produce additional insight about legislators without requiring additional data collection.
	
	In this paper I first describe the current state of ideal point modeling with an attention to the growing awareness of the problem of absences and abstentions. I then present the model formally and offer simulation results to verify the model's performance. Then I examine two different empirical applications of the model, one drawn from data from the United States Congress and the second from the parliament in Tunisia's transitional democracy.
	
	\section*{Missing Data in Ideal Point Modeling}
	
	Following the pioneering work of \textcite{poole1997} and \textcite{jackman2004}, ideal point models have become a standard feature of the analysis of legislators and increasingly other political actors. The canonical ideal point model expresses legislator preferences as distances from a latent position in an $n$-dimensional policy space \parencite{enelow198} where the distances can be calculated either as Euclidean distances (IRT) or using the Normal distribution (i.e., the NOMINATE models), although both of these approaches tend to offer very similar estimates  \parencite{carroll2009}. A fully Bayesian analysis such as \textcite{jackman2004} is able to analyze virtually all bills, but whether done in a frequentist or Bayesian framework, the focus has remained on ideal points as representations of yes/no votes.
	
	Recently there has been criticism of these approaches because the ideal points that are estimated may not be an accurate measure of a legislator's ``true" ideological stance \parencite{krehbiel2014,Caughey2016,brauninger2016}. Statistically speaking, the ideal points are the dimension of variance that best explains the observed outcomes, so there is no easy way to know whether the ideal points refer to a legislator's ideological inclinations or more to his or her political strategies. For the model I present, however, this latter interpretation is more helpful because it makes it clear why data on absences and abstentions would be important to include. To the extent that voting can be represented in a unidimensional space, this space should be able to account for the full range of legislative behavior. The ideal point may properly refer to a legislator's firm political convictions or it may reflect their desire to appear more moderate/liberal/conservative for tactical reasons. Regardless, these ideal points are by definition the lowest-error explanation of observed behavior, and for that reason they are worth studying, even if as latent variables we cannot fully explain exactly what they represent.
	
	Building on this work, political scientists have begun to look at alternative sources of data about ideal points beyond the collection of yes and no votes. A growing literature uses data generated outside of the legislature, whether via Twitter \parencite{barbera2015}, through political donations \parencite{bonica2014}, or through the legislator's prior history as a state representative \parencite{shor2011}. Most recently, \textcite{brauninger2016} and \textcite{rosas2015} have proposed methods for including abstentions and absences in parliamentary rollcall voting, while \textcite{powell2016} has presented new data on Congressional absences that shows how different types of absences may provide varying signals of legislator partisanship.
	
	None of these papers, however, attempts to directly incorporate absences or abstentions into the afore-mentioned ideal point models, which usually code absences and abstentions as missing data. \textcite{rubin2002} argue that for missing data to be ignorable, the probability that an observation is missing must not depend on the value of the observation. \textcite{rosas2015} is the first to apply this theoretical approach to rollcall voting data via an imputation model of legislative behavior framed in terms of disagreement between a party member and the party's official position.  As \textcite{rosas2015} point out, in the case of rollcall vote data, the assumption that absences and abstentions can be thought of as a plausibly random distribution yes/no votes is unlikely to be true. In \citeauthor{rosas2015}'s model, yes and no votes are imputed based on whether a party member is in agreement with his or her party given the assumption that, conditional on knowing whether a party member is in disagreement, the ensuing decision to vote yes or no only depends on observed ideal points \parencite{rubin2002}, and hence is imputable. This model provides a compelling account of how a legislator may decide to abstain in a vote; however, the model's applicability is also limited to this situation. In addition, \citeauthor{rosas2015} collapse both absences and abstentions into a single category because in their framework these actions are equal signals of party disagreement.
	
	By comparison, the model that I propose should apply to abstention and absence data more broadly, although it does not provide special insight into intra-party dynamics. The main difference is that I do not attempt to impute observations. 
	
	
	
	\section*{Absence-Inflated Hurdle Model}
	
	To handle the full range of strategic outcomes in legislative data, I propose embedding two IRT equations inside of a hurdle model. A hurdle model, which is similar to a zero-inflated model, accounts for one-way censoring in data by separately modeling a probability that an observation belongs to either the hurdle or to the full IRT model. 
	
	I begin with an IRT ordered logit model, which is known as a rating-scale model in the IRT literature.\footnote{A rating-scale model has fixed cutpoints for the entire data and matches the parameterization of the commonly-used ordered logit model. The graded-response model, on the other hand, estimates separate cutpoints for each bill. While it is an interesting extension of the basic model, it is not considered in this paper, although it can be estimated in the \texttt{idealstan} R package.} The $K$ category vote outcome $_k$ can be modeled as a likelihood function $L(\cdot)$ of $I$ legislator ideal points $x_i$, $J$ bill discrimination parameters $\beta_j$ and bill intercepts $\alpha_j$. Each equation is put inside $\zeta(\cdot)$, which represents the logit function. Formally, the full likelihood is:
	
					\[
	L(\beta,\alpha,x|Y_{k}) = \prod_{n}^{i=1} \prod_{m}^{j=1}
	\begin{cases} 
	1 -  \zeta(x_{i}'\beta_j - \alpha_j - c_1) & \text{if } K = 0 \\
	\zeta(x_{i}'\beta_j - \alpha_j - c_{k-1}) - \zeta(x_{i}'\beta_j - \alpha_j - c_{k})       & \text{if } 0 < k < K, \text{ and} \\
	\zeta(x_{i}'\beta_j - \alpha_j - c_{k-1}) - 0 & \text{if } k=K
	\end{cases}
	\]
	
	The use of $K$ ordered outcomes allows the model to incorporate abstentions as a third category between yes and no votes. Because the cutpoints are estimated from the data, the actual width of the utility apportioned to voting in each category will vary from dataset to dataset. In the Bayesian framework with priors on variables described later, there is no risk of cutpoint collapse because any bills with zero abstentions will simply default to the prior.
	
	The bill intercepts also have a slightly different interpretation in this model. The intercept for each bill is actually $\alpha_j + c_k$ for all $K$ so that in effect the bill has an intercept for each category of vote outcome. Otherwise, however, the estimated parameters are interpreted identically to the standard IRT model \parencite{jackman2004}. The $x_i$ represent legislator ideal points and the bill midpoints (i.e., the line of indifference for each vote outcome) are equal to $\frac{\alpha_j + c_{k-1}}{\beta_j}$. Thus for a rollcall voting model with three distinct outcomes, yes, no and abstain, there will be two lines of indifference (or equiprobability contours) for each bill that separate legislators into groups that are more or less likely to vote in each vote category. For the binary case in which $K=2$, the cutpoints collapse into the bill intercepts and a standard binary logit model is estimated.
	
		 \[
	L(\beta,\alpha,X,Q,\gamma,\omega,\phi|Y_{k},Y_{r}) = 
	\prod_{n}^{i=1} \prod_{m}^{j=1}
	\begin{cases}
	\zeta({x_{i}'\gamma_j - \omega_j + \phi q_i}) & \text{if } r=0, \text{ and} \\
	(1-\zeta({x_{i}'\gamma_j - \omega_j + \phi q_i}))L(\beta,\alpha,X|Y_{k1}) & \text{if } r=1
	\end{cases}
	\]
	
	\section*{Simulation Results}
	
	\section*{Empirical Applications}
	
	\subsection*{US Congress}
	
	\subsection*{Tunisian Parliament}
	
	\section*{Discussion}
	
	\section*{Conclusion}
	
	\section*{Appendix}
	
	\section*{References}
	
	
	
\end{document}